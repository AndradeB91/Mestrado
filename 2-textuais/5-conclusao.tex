\chapter{Conclusão e Trabalhos Futuros}
\label{chap:conclusao-e-trabalhos-futuros}

\section{Principais Contribuições} 
Esse trabalho apresentou uma técnica de otimização de malhas para remoção de ruídos de filtragem bilateral com pré-processamento. A técnica proposta apresenta bons resultados para qualquer tipo de modelo com ruído e ela possui um passo de pré-processamento que tem o objetivo de contribuir na otimização de, principalmente, modelos com ruídos que possuem qualidade de elementos excessivamente ruim.

Ao final deste trabalho foram apresentados, com ajuda de figuras, histogramas e tabelas, os resultados obtidos pela técnica proposta. Foi também mostrada a importância de melhorar a qualidade dos elementos de uma malha anteriormente à otimização propriamente dita. Conjuntamente, comparações foram apresentadas para mostrar o melhor desempenho da técnica proposta com o estado da arte em algoritmos de otimização de malhas para remoção de ruídos, em especial as técnicas descritas em \cite{zhang2015guided, sun2007fast, zheng2011bilateral}.

O trabalho proposto pode ser dividido basicamente em três etapas. A primeira consiste na análise e seleção de \textit{patches} de elementos. A segunda constitui-se do \textit{remalhamento} dos \textit{patches} selecionados, convertendo essas regiões em áreas com uma melhor qualidade de elementos. A terceira e última etapa consiste na filtragem das normais da malha seguido da atualização de seus vértices. Essa última etapa é reproduzida iterativamente até um resultado satisfatório (cerca de 10 a 15 iterações dependendo do modelo).

As principais contribuições deste trabalho são a apresentação de uma nova técnica de otimização de malhas para remoção de ruídos, que também melhora a qualidade de seus elementos, e a confirmação da importância que a melhoria local da qualidade dos elementos da malha produz no resultado de uma técnica de otimização de malhas com intuito de remoção de ruídos.

\section{Trabalhos Futuros}

A técnica aqui apresentada possui três etapas principais. Na primeira etapa foi utilizado um julgamento visual por parte do usuário, onde ele escolhe as regiões a serem \textit{remalhadas}. Uma investigação mais aprofundada deverá ser realizada no intuito da automatização desse processo, onde métricas de qualidade de triângulos poderão ser usadas para saber quais regiões da malha entrarão no processo de \textit{remalhamento}. 

Na segunda etapa, o algoritmo de \cite{miranda2009surface} foi utilizado. Algumas outras opções como \cite{attene2013polygon, zhao2007robust} também poderão ser testadas, pois talvez possam produzir resultados ainda melhores para reconstrução de malhas. No entanto, no trabalho proposto há a existência de uma malha suporte, fazendo com que o \textit{remalhamento} siga exatamente essa malha suporte, sendo portanto muito improvável que os resultados utilizando as técnicas de \cite{attene2013polygon, zhao2007robust}, que predizem por onde a reconstrução deve seguir, superem os resultados já apresentados no Capítulo \ref{chap:exemploseresultados}. Um estudo mais detalhado, com comparações e implementações, deve ser realizado para este passo.

Na última etapa, um algoritmo de filtragem das normais e atualização de vértices é proposto baseado em \cite{zhang2015guided}. Mais algumas comparações podem ser realizadas utilizando outras técnicas de otimização de malha neste passo, estendendo, dessa forma, a técnica proposta a uma técnica geral de pré-processamento para otimização de malhas, onde qualquer algoritmo de otimização pode ser configurado. Essa extensão não é algo trivial, necessitando de estudos mais aprofundados.

