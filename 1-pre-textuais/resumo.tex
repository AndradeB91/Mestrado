Modelos de alta resolução, baseados em malha, são principalmente gerados por scanners ópticos 3D. Esta abordagem de aquisição de malha introduz ruídos e irregularidades, de várias fontes distintas, na malha, fazendo com que ela seja inutilizável em aplicações práticas onde modelos de alta qualidade são necessários. Otimizar malhas com ruído, enquanto preservam-se suas características geométricas, é uma tarefa desafiadora. Muitos métodos anisotrópicos de otimização de malha, inspirados nos conceitos de Processamento de Imagens, foram propostos na década passada tentando resolver este problema. Eles obtêm bons resultados na maioria dos casos, falhando em otimizar malhas com grande quantidade de ruído ou com amostragem de elementos extremamente ruins. É apresentada, neste trabalho, uma técnica de filtragem de normais com pré-processamento baseada no filtro bilateral conjunto (\textit{joint bilateral filtering}), a ser aplicada, principalmente, em malhas com amostragem de elementos extremamente ruim. A técnica consiste, basicamente, em uma abordagem de dois passos: primeiramente, a qualidade dos elementos da malha é aprimorada detectando e remalhando regiões com amostragem de elementos ruim; consecutivamente, é aplicada a filtragem bilateral conjunta nas normais das faces da malha seguida de uma atualização da posição dos vértices de acordo com o novo campo de normais filtradas. É mostrado que, ao empregar esse novo método, podem-se produzir melhores resultados ao otimizar malhas com amostragem de elementos irregular. A eficácia do método proposto é validada através de experimentos considerando diferentes modelos baseados em malha. 
 

% Separe as palavras-chave por ponto
\palavraschave{Processamento de geometria. Otimização de malhas. Remoção de ruídos em malhas. Filtragem de malha.}

