High-resolution mesh models are mainly generated by advanced 3D optical laser scanners. This mesh acquisition approach introduces noise from various sources and irregularities on the mesh, making it unusable in practical applications where high quality models are required. Optimizing a noisy mesh, while preserving its geometric features, is a challenging task. Many anisotropic mesh optimization methods, inspired by the concepts in image processing, were proposed in the past decade attempting to solve this problem. They attain fine results in most cases, failing on optimizing meshes with large amounts of noise or with extremely irregular sampling. We present, in this work, a preprocessed normal filtering technique based on the joint bilateral filtering to be applied mainly on meshes with extremely irregular sampling. It is a two-stage approach: first, we improve the quality of the mesh elements by detecting and re-meshing irregular sampling regions; subsequently, we apply a joint bilateral filtering to the face normals followed by the vertex position update according to the new face normals field. We show that, by employing this new method, we can produce better results optimizing such irregular sampling meshes. The effectiveness of the proposed method is validated through experiments evaluation that considers different mesh models.

% Separe as Keywords por ponto
\keywords{Geometry processing. Mesh optimization. Mesh denoising. Mesh filtering.}